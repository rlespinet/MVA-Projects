\documentclass{article}
%%%%
% Provide the command \fpeval as a copy of the code-level \fp_eval:n.
\usepackage{expl3}[2012-07-08]
\ExplSyntaxOn
\cs_new_eq:NN \fpeval \fp_eval:n
\ExplSyntaxOff
%%%%
\usepackage[english]{babel}
\usepackage[utf8]{inputenc}
\usepackage[margin=1.2in]{geometry}
\usepackage{amsmath}
\usepackage{amsthm}
\usepackage{amsfonts}
\usepackage{amssymb}
\usepackage[usenames,dvipsnames]{xcolor}
\usepackage{graphicx}
\usepackage[siunitx]{circuitikz}
\usepackage{tikz}
\usepackage[colorinlistoftodos, color=orange!50]{todonotes}
\usepackage{hyperref}
\usepackage[numbers, square]{natbib}
\usepackage{fancybox}
\usepackage{epsfig}
\usepackage{soul}
\usepackage[framemethod=tikz]{mdframed}
\usepackage{enumitem}
\usepackage{subcaption}
\usepackage{titlesec}
\usepackage{booktabs}
\input{macros}

%% \setlength{\parskip}{\baselineskip}%
\renewcommand\thesection{\Roman{section}}
\renewcommand\thesubsection{\Roman{section}.\Alph{subsection}}
\titlespacing{\section}
              {0pt}{1\baselineskip}{1\baselineskip}
\titlespacing{\subsection}
              {0pt}{0.8\baselineskip}{0.8\baselineskip}

%%%%%%%%%%%%%%%%%%%%%%%%%%%%%%%%%%%%%%%%%%%
% Header
%%%%%%%%%%%%%%%%%%%%%%%%%%%%%%%%%%%%%%%%%%%

\renewcommand{\assignmenttitle}{Assignment 1: Instance-level recognition}
\renewcommand{\studentname}{Rémi Lespinet}
\renewcommand{\email}{remi.lespinet@ens-paris-saclay.fr}

%%%%%%%%%%%%%%%%%%%%%%%%%%%%%%%%%%%%%%%%%%%
% Syntax for using figure macros:
%%%%%%%%%%%%%%%%%%%%%%%%%%%%%%%%%%%%%%%%%%%

% \singlefig{filename}{scalefactor}{caption}{label}
% \doublefig{\subfig{filename}{scalefactor}{subcaption}{sublabel}}
%           {\subfig{filename}{scalefactor}{subcaption}{sublabel}}
%           {global caption}{label}
% \triplefig{\subfig{filename}{scalefactor}{subcaption}{sublabel}}
%           {\subfig{filename}{scalefactor}{subcaption}{sublabel}}
%           {\subfig{filename}{scalefactor}{subcaption}{sublabel}}
%           {global caption}{label}
%
% Tips:
% - with scalefactor=1, a single figure will take the whole page width; a double figure, half page width; and a triple figure, a third of the page width
% - image files should be placed in the image folder
% - no need to put image extension to include the image
% - for vector graphics (plots), pdf figures are suggested
% - for images, jpg/png are suggested
% - labels can be left empty {}

%%%%%%%%%%%%%%%%%%%%%%%%%%%%%%%%%%%%%%%%%%%
% Beginning of assignment
%%%%%%%%%%%%%%%%%%%%%%%%%%%%%%%%%%%%%%%%%%%
\begin{document}
\maketitle

\section{Sparse features for matching specific objects in images}

\subsection{SIFT features detections}

\question{QIA.1: (i) Why is it important to have a similarity co-variant feature detector? (ii) How does this affect the descriptors computed at these detections? (iii) How does this affect the matching process?}

\begin{enumerate}
\item For a wide range of application, it's really important to have a
  feature detector match the same features when there's rotation
  translation or scale change.  In our case, %% we want to recognize
  %% objects (that is compute feature descriptors and search for similar
  %% descriptors in our database of labeled images). We
  we want our ability
  to recognize objects to be independant from their position in the
  scene, in particular we want to find the same features for an
  object regardless of
  \begin{itemize}
    \item if its position in the image space has been translated : translation invariance
    \item if it's close or far from the viewpoint : scale invariance
    \item if the object or the image has been rotated in the image space : rotation invariance
  \end{itemize}

  %% This question is very general, I'll answer it in the context of
  %% object recognition, as there might be cases where we don't need scale
  %% invariant descriptors for example (satellite images

  %% Depending on the domain of application, we want to match objects
  %% in images

  %% The goal of feature based detection is to extract descriptors
  %% that locally represent the image, so that we can use these
  %% descriptors to recognize objects in the scene, in our case
  %% query labeled images that present similar descriptors, to get
  %% the object represented in the image


\item In order for the co-variant feature detector to make sense, the
descriptors computed at these detections must also be invariant
by rotation, scale and translation.

\item The matching process is simplified, since the descriptors are
  invariant by rotation scale and translation, we just have to match
  descriptors that are close to each other (for example in the
  euclidan distance sense). This would yield a match even if one image
  is rotated, scaled or translated compared to the other.

\end{enumerate}

\question{QIA.2: Show the detected features in the two images for three different values of the peakThreshold option}

The detected features for three values of the \emph{peakThreshold}
parameters ($0.0005$, $0.001$ and $0.005$) are represented on
figure~\ref{fig:features-im1} for the image 1
(\emph{all\_souls\_000002.jpg}) and on figure~\ref{fig:features-im2}
for the image 2 (\emph{all\_souls\_000015.jpg})

\triplefig{\subfig{features-im1-0-0005.png}{1}{$peakThreshold = 0.0005$}{fig:im1-0.0005}}
          {\subfig{features-im1-0-001.png}{1}{$peakThreshold = 0.001$}{fig:im1-0.001}}
          {\subfig{features-im1-0-005.png}{1}{$peakThreshold = 0.005$}{fig:im1-0.005}}
          {Feature detectors with three different values of $peakThreshold$ for image1}{fig:features-im1}

\triplefig{\subfig{features-im2-0-0005.png}{1}{$peakThreshold = 0.0005$}{fig:im2-0.0005}}
          {\subfig{features-im2-0-001.png}{1}{$peakThreshold = 0.001$}{fig:im2-0.001}}
          {\subfig{features-im2-0-005.png}{1}{$peakThreshold = 0.005$}{fig:im2-0.005}}
          {Feature detectors with three different values of $peakThreshold$ for image2}{fig:features-im2}

\question{QIA.3: Note the change in spatial density of detections across images, for a given value of peakThreshold. (i) Is the density uniform? If not, why? (ii) Which implications for image matching can this have? (iii) How can it be avoided ?}

\begin{enumerate}

\item The spatial density is not uniform between images, the value of
the parameter "peakThreshold" determines the minimum value of the
difference of gaussian that is accepted.  Zones of the image that have
more details will have more keypoints, moreover, this is not invariant
by non linear contrast change, so zones that are more contrasted will
also have a higher number of contrast points.

\item This will have an impact of matching, if the images that we want to
match have very different contrasts, the same value of the peakThreshold
can lead to a very different number of points, which is an additionnal
difficulty in the matching process

\item There are two way to avoid this issue, the first is to process the
image before finding the keypoints (for example histogram
equalization), and the second is to filter the keypoints after this
operation.

\end{enumerate}


\subsection{SIFT features descriptors and matching between images}

\question{QIB.1: Note the descriptors are computed over a much larger region (shown in blue) than the detection (shown in green). Why is this a good strategy?}

The detection part of the SIFT algorithm gives us a point, and a
scale, the idea then is to compute the descriptors over a region
around the keypoints found. This is a good strategy, because it makes
the the descriptors robust against small changes in the local geometry
around the detected keypoint.


\question{QIB.2: Examine carefully the mismatches and try to understand the different causes of mismatch. (i) In your report, present at least 3 of them and explain why the mistakes are being made. For example, is the change in lighting a problem? (ii) What additional constraints can be applied to remove the mismatches? }

\begin{enumerate}

\item The figure~\ref{fig:sift50matches} presents $50$ matches given
  by the nearest neighbor methods for the two images \emph{all\_souls\_000002.jpg} and
  \emph{all\_souls\_000015.jpg} for \emph{peakThreshold = 0.005}
  The first thing we can notice is that there's a lot of mismatch

  \begin{itemize}
  \item In the right image of figure~\ref{fig:sift50matches} we see
    that half of the building is covered with shade, and we see that
    none of the 3D points at this location in image 1 have been
    matched at the right place, this is because the whole shaded
    region has a lower contrast, which implies that there are no
    keypoints in this region as explained in the question QIA.3.
    This is a first source of mistake.

  \item We also see that a window in the image2 reflects all the sun
    which results in saturation, and this window appears completely
    white. There's a keypoint in the image 1 which would normally have
    a correspondance on this location in the image 2, but due to this
    light effect, a mistake is made for this point. The light is a huge
    problem, since it induces effects that are highly not linear
    (specularities, shades, ...).

  \item We see that the building has repeated patterns, windows,
    ornements are all very similars, this means, that for the
    descriptors that are located in a window will be really close to
    each other (they will be located in a small ball in
    $\mathbb{R}^{128}$), moreover taking a picture is a process that
    is noisy in essence (not even talking about the light conditions
    and the jpg compression which induces more noise). this makes the
    nearest neighbor algorithm retrieve a wrong descriptor (for
    example if the descriptor is located on a window, it retrieves
    another window).

  \end{itemize}

\singlefig{sift50matches}{0.8}{Representation of $50$ matches  given
  by the nearest neighbor methods for the two images \emph{all\_souls\_000002.jpg} and
  \emph{all\_souls\_000015.jpg} for $peakThreshold = 0.005$.}{fig:sift50matches}

\item We could apply constraints that impose all the matches to be
  coherent with each other (geometric constraints), e.g. find a subset
  of matches that are coherent in the geometrical sense. We could also
  drop the matches if there's too much uncertainty in this match in
  some sense (this is what we are going to do in the next section)

\end{enumerate}

\subsection{Improving SIFT matching (i) using Lowe’s second nearest neighbour test}

\question{QIC.1: Illustrate and comment on improvements that you can achieve with this step. Include in your report figures showing the varying number of tentative matches when you change the nnThreshold parameter.}

The first thing we can notice, is that it reduces drastically the
number of points

\begin{center}
\begin{tabular}{|c|c|c|}
  \hline
  nnThreshold & points & removed (\%) \\
  \hline
  1 (unchanged) & 458 & $0\%$ \\
  0.8 & 96 & $79.04\%$ \\
  0.6 & 35 & $92.36\%$ \\
  0.4 & 14 & $96.94\%$ \\
  \hline
\end{tabular}
\end{center}

The figure~\ref{fig:lowe-test} shows the different matches kept for 3
values of $nnThreshold$, we can see that for $nnThreshold = 0.6$
there's 35 points which are almost all good matches.  For $nnThreshold
= 0.4$, there's only 14 points left but they all are good matches.

\triplefign{\subfig{lowe_crit_0_8.png}{0.9}{$nnThreshold = 0.8$ (96 points kept over 458) }{fig:lowe-0.8}}
           {\subfig{lowe_crit_0_6.png}{0.9}{$nnThreshold = 0.6$ (35 points kept over 458)}{fig:lowe-0.6}}
           {\subfig{lowe_crit_0_4.png}{0.9}{$nnThreshold = 0.4$ (14 points kept over 458)}{fig:lowe-0.4}}
           {Lowe's second nearest neighbour test for different values of $nnThreshold$}{fig:lowe-test}

\newpage

The figure~\ref{fig:lowe-removed} shows the $42$ points removed when
we take $nnThreshold = 0.99$, these are the $42$ points whose nearest
neighbor ratio are closest to $1$. We can see that a lot of mismatch
are removed, but some good matches are also discarded with this
criterion.

\singlefig{lowe_crit_0_99_removed.png}{0.6}{Representation of the $42$ points removed when taking $nnThreshold = 0.99$}{fig:lowe-removed}

We see in these examples that the method seems to work very well, at
the cost of discarding a lot of points so there's a tradeoff between
the number of points and the quality of the match.




\subsection{Improving SIFT matching (ii) using a geometric transformation}

\question{QID.1: Work out how to compute this transformation from a single correspondence. Hint: recall from Stage I.A that a SIFT feature frame is an oriented circle and map one onto the other.}

%% A sift keypoint consists of a position and a scale. Moreover at a
%% keypoint, a descriptor contains the rotation information, so, a frame

Recall the QI.A, each SIFT frame consists of a position ($t_x$,
$t_y$), a scale $s$, and a rotation $\theta$. If we have a match, we
have 2 SIFT frame $(t^{(1)}_x, t^{(1)}_y, s_1, \theta_1)$ and
$(t^{(2)}_x, t^{(2)}_y, s_2, \theta_2)$

The relation is given by
\begin{equation*}
  \left[
  \begin{array}{l}
    x' \\
    y'
  \end{array}
  \right] =
  \dfrac{s_2}{s_1}
  \left[
  \begin{array}{lr}
    cos(\theta_2 - \theta_1) & -sin(\theta_2 - \theta_1) \\
    sin(\theta_2 - \theta_1) & cos(\theta_2 - \theta_1) \\
  \end{array}
  \right]
  \left[
  \begin{array}{l}
    x \\
    y
  \end{array}
  \right] +
  \left[
  \begin{array}{l}
    t^{(2)}_x - t^{(1)}_x \\
    t^{(2)}_y - t^{(1)}_y
  \end{array}
  \right]
\end{equation*}

\question{QID.2: Illustrate improvements that you can achieve with this step.}

The figure~\ref{fig:geom-verif} presents the matches obtained after
geometric verification, we can verify that all matches are correct.
This method is a great improvement, it allows to keep the largest
number of matches that are coherent with the hypothesis of similarity
change between the two images. This is a good approximation in our
case since the angles between the viewpoint and the planes of the
scene doesn't change much between the two images.

\singlefig{geom_verif_0_005.png}{0.6}{Representation of the matches obtained by geometric verification ($peakThreshold = 0.005$ and Lowe's test is not used)}{fig:geom-verif}

\section{Affine co-variant detectors}

\question{QII.1: Include in your report images and a graph showing the number of verified matches with changing viewpoint. At first there are more similarity detector matches than affine. Why?}


We see that the similarity co-variant detector (SIFT) behaves
well between images 1 and 2 and between images 1 and 3, this is
because the perspective transformation is not extreme, and it is well
approximated by a similarity. The figure~\ref{fig:affine-1-3} compares
the number of matches between similiarity and affine co-variant
detectors.

\doublefig{\subfig{similarity_1_3.png}{1}{similarity co-variant detector}{fig:similarity-1-3}}
        {\subfig{affinity_1_3.png}{1}{affine co-variant detector}{fig:affinity-1-3}}
        {Comparison of the number of matches between similarity and affine co-variant detectors for low perspective (images 1 and 3)}{fig:affine-1-3}

When the perspective between the two images starts to be more
aggressive, the descriptors of the similarity co-variant detector
fails to represent the local geometry accurately, which leads to a
very low number of good matches after geometric verification.  (In the
exterme case of image \emph{graf/img6.png} there's only 6 matches
left, and only 3 of them are correct). The affine co-variant detector
on the other hand, still provides a good number of matches even in
extreme perspective conditions (see figure~\ref{fig:affine-1-5}).

\doublefig{\subfig{similarity_1_5.png}{1}{similarity co-variant detector}{fig:similarity-1-5}}
        {\subfig{affinity_1_5.png}{1}{affine co-variant detector}{fig:affinity-1-5}}
        {Comparison of the number of matches between similarity and affine co-variant detectors for high perspective (images 1 and 5)}{fig:affine-1-5}

The figure~\ref{fig:similarity-vs-affine} shows the evolution of the
number of matches as the change in perspective increases for the
similarity co-variant and affine co-variant descriptor. We see that
the number of matches in the similarity co-variant detector decrease
more rapidly than the affine co-variant detector as the perspective
becomes more and more extreme.

\singlefig{similarity_vs_affine.png}{0.55}{Representation of the number of matches as the function of the perspective (abscissa i corresponds to the number of matches between image 1 and image i)}{fig:similarity-vs-affine}

We notice that the similarity co-variant detector produces more matches
when the perspective is small, the first thing that we can see is that
the \emph{getFeatures} function produces less keypoints (see table below)
but this cannot explain this behavior because the number of extra points
found is negligeable (see the table below).

\begin{center}
  \begin{tabular}{| c | c | c |}
    \hline
    Image & Similarity co-variant & Affine co-variant \\
    \hline
    1 & 4474 & 4097 \\
    2 & 4837 & 4805 \\
    3 & 5158 & 5064 \\
    4 & 5222 & 5046 \\
    5 & 5803 & 5177 \\
    6 & 5660 & 4929 \\
    \hline
  \end{tabular}
  \captionof{table}{Comparison of the number of keypoint produced with
    each detector for each image in our persepective set}
\end{center}

By looking closely at the matches produced by the similarity
co-variant detector after the geometric verification, we see that
there's a lot of mismatches, this is due to the fact that we try to
find a similarity between those two images, but the real underlying
transformation is not a similarity, so the \emph{RANSAC} algorithm
finds the best affinity that could best explain the matches, and it's
parameters $(t_x, t_y, s, \theta)$ are erroneous. When
computing the parameters for each match in the second part of the
\emph{RANSAC} algorithm, we would have points whose $\theta$ and $s$
are really close to the original, but whose $t_x$ and $t_y$ are
really far, and these points would not be removed by the threshold.


\section{Towards large scale retrieval}

\subsection{Accelerating descriptor matching with visual words}

\question{QIIIA.1: In the above procedure the time required to convert the descriptors into visual words was not accounted for. Why the speed of this step is less important in practice?}

In practice this step is less important because we want to match an
image against a database, and the step of converting thee descriptors
to visual words can be done as a preprocessing step, e.g we only need
to do it once for all images at the beginning, we can then use the
visual words for every query without having to recompute them (except
when we need to add images)

\question{QIIIA.2: What is the speedup in seconds in searching a large, fixed database of 10, 100, 1000 images? Measure and report the speedup in seconds for 10, 100 and 1000 images.}

Suppose that we have extracted $n$ keypoints from image 1 and $m$
keypoints from image 2, and suppose $m < n$

With this method, we only need to compute the word indices for the 2
images, this gives us a vector consisting of the id of each word
associated with a descriptor. The matches can then be computed very
efficiently in $O(n)$ by computing an histogram ($O(m)$) and iterating
over the words indices of image 1 and lookup in the computed histogram
($O(n)$).

In the previous method, we had to find the 2 nearest neighbor for each
descriptor (This is difficult in high dimension (128 for sift), using
approximate nearest neighbor with LSH, the query complexity is about
$O(d m^{1/(1 + \epsilon)^2})$ with $d$ being the dimension (128 for
sift) and $1 + \epsilon$ the factor between the real nearest neighbour
and an accepted answer. We then have to iterate over all descriptors to
have all matches, this lead to $O(d n m^{1/(1 + \epsilon)^2})$ (There's also
a preprocessing time that I dont put in the equation)

%% For one image, the matches with the approximate nearest neighbor
%% method are computed in about $0.0173s$ on my personal computer

The following table report the time in seconds for the nearest
neighbor method and for the visual words method

\begin{center}
  \begin{tabular}{| c | c | c |}
    \hline
    database size & ANN & Visual words \\
    \hline
    1    & 0.0253  & 0.000602 \\
    10   & 0.2492  & 0.004719 \\
    100  & 2.2120  & 0.042452 \\
    1000 & 22.1203 & 0.424516 \\
    \hline
  \end{tabular}
  \captionof{table}{Comparison of the time (in second) needed to match
    descriptors from an image to images in the database with the 2 methods}
\end{center}

To compute this table, I only measure the cumulated time spent on the
\emph{findNeighbor} for the ANN method and \emph{matchWords} for the
visual words. For the ANN method, this suppose that we have
precomputed the features for all images in the database and that they
fit in memory. For the Visual words method, this suppose that the
kdtree has already been built (which is the usual case).
each cells is the mean of 5 run of the algorithm on the corresponding
database. For the 1000 size database, I added artificially pictures
by duplication (because the database was not big enough)

\subsection{Searching with an inverted index}

\question{QIIIB.1: Why does the top image have a score of 1?}

The top image has a score of 1 because it is exactly the image
that we have queried (e.g we query an image that is in the database)

\question{QIIIB.2: Show the first 25 matching results, indicating the correct and incorrect matches. How many erroneously matched images do you count in the top results?}

The figure~\ref{fig:search-index-25} shows the 25 images with the
highest number of visual words in common with the query image
representation.  As explained in the previous question, the queried
image is the the one in the top-left corner (with score 1)

\singlefig{search_index_25.png}{1}{top 25 images with the highest number
  of visual words in common with the query image (top-left image) in
  the database \emph{oxbuild\_lite} }{fig:search-index-25}

We see that 9 images over the 25 images returned are correct (they
represent the same building as the queried image) There are 16 images
that are incorrect. We also notice that there are images of this
building that do not appear in these 25 images, and that the best
match (excluding the image itself) is incorrect.

However we can notice that the mismatched images, look very similar to
the queried image (in particular, the Radcliffe Camera that is present
11 times has a very similar structure with pillars, statues, molding,
and the materieals are very similars (same colors, some parts have
bricks texture,...))


\subsection{Geometric rescoring}


\question{QIIIC.1: Why is the top score much larger than 1 now?}

The idea is to use the 25 images obtained via the precedent algorithm,
and add a geometric verification step. The score that we see in
figure~\ref{fig:search-index-25-reorder} is simply the number of
matches after this geometrical verification step, this explains why
the top score is much larger than 1, it corresponds to the number of
geometrically verified matches between the image and itself (here
1241).


\singlefig{search_index_25_reorder.png}{1}{top 25 images with the
  highest number of visual words in common with the query image
  (top-left image) reordered according to their number of
  geometrically verified matches with the query image}{fig:search-index-25-reorder}


\question{QIIIC.2: Illustrate improvements of retrieval results after
  geometric verification.}

The figure~\ref{fig:search-index-25-reorder} presents the 25 images
with the highest number of visual words in common with the query
image, with their new score (the number of geometrically verified
matches). The images have been re-ordered according to this new score.

Re-ordering according to this new score is a good improvement. The 9
good images are propulsed to the top of the classement.  We can also
notice there's a gap between the last good image (score $37$ and the
first wrong image $10$), and all mismatches have a very low score (less
than $10$), this could be used to remove this images from the search.

We notice that there are images representing the same building that do
not appear in this set of 25 images, which would have obtained a higher
number of geometrically verified matches than some of the images in
this set (namely the ones that are wrong), but the whole point of doing
the search this way is to avoid the computationally expensive cost of
geometric verification for all images. So this is a good
approximation, altough it could discard the best matching image in the
dataset if there are many other images that have a huge number of
local patches in common with the query image, it's unlikely that it
happens.


\section{Large scale retrieval}

\question{QIV.1: How many features are there in the painting database?}

There's $100 000$ visual words in this database.

\question{QIV.2: How much memory does the image database take?}

The file takes 191 906 193 bytes on disk.

\question{QIV.3: What are the stages of the search? And how long does each of the stages take for one of the query images?}

%% \begin{enumerate}[label=\arabic*)]

%% \item Compute images descriptors
%% \item Quantify descriptors using the database and get a visual words
%%   index per descriptor
%% \item Create the histogram with visual words indices
%% \item For all images in the database compute a similarity score between
%%   this image's histogram and the histogram of the queried image
%% \item Sort the images according to their similarity score in
%%   descending order
%% \item For the first $N$ images (here $N = 100$)
%%   \begin{enumerate}
%%     \item Search for a good subset of matches that are geometrically
%%       coherent with the queried image (RANSAC) (Geometric
%%       verification)
%%     \item Assign this image the number of geometrically verified
%%       matches as geometric score
%%   \end{enumerate}
%% \item Sort the first $N$ images according to their geometric score
%%   (descending order)
%% \item Return the first $M < N$ images

%% \end{enumerate}

\begin{enumerate}[label=\arabic*)]

  \item \textbf{Images features}

\begin{enumerate}[label=\arabic*)]
\item Compute images descriptors
\item Quantify descriptors using the database and get a visual words
  index per descriptor
\item Create the histogram with visual words indices
\end{enumerate}

  \item \textbf{Inverted index}

\begin{enumerate}[label=\arabic*)]
\item For all images in the database compute a similarity score between
  this image's histogram and the histogram of the queried image
\item Sort the images according to their similarity score in
  descending order
\end{enumerate}

\clearpage

\item \textbf{Geometric verification}

\begin{enumerate}[label=\arabic*)]
\item For the first $N$ images (here $N = 100$)
  \begin{enumerate}
  \item Search for a good subset of matches that are geometrically
    coherent with the queried image (RANSAC) (Geometric
    verification)
  \item Assign this image the number of geometrically verified
    matches as geometric score
  \end{enumerate}
\item Sort the first $N$ images according to their geometric score
  (descending order)
\item Return the first $M < N$ images

\end{enumerate}

\end{enumerate}

    %% mistery-painting1.jpg & 0.181246 & 0.020612 & 0.241689 \\
    %% mistery-painting2.jpg & 0.326188 & 0.020653 & 0.382553 \\
    %% mistery-painting3.jpg & 1.442933 & 0.022114 & 1.360761 \\

I executed the algorithm 10 times on each mystery painting. The
following table presents the mean time spent on the different parts.
The difference in execution time is due to the images having very
different sizes.

\begin{center}
  \begin{tabular}{| c | c | c | c |}
    \hline
    Images & Images features & Inverted index & Geometric verification \\
    \hline
    mistery-painting1.jpg (150x182) & 0.181 & 0.0206 & 0.242 \\
    mistery-painting2.jpg (286x232) & 0.326 & 0.0207 & 0.383 \\
    mistery-painting3.jpg (690x540) & 1.443 & 0.0221 & 1.361 \\
    \hline
  \end{tabular}
  \captionof{table}{Comparison of the time spent in the different part
    of the algorithm (in second)}
\end{center}


%% \begin{enumerate}[label=\arabic*)]

%% \item Image features
%%   \begin{enumerate}
%%   \item Compute images descriptors
%%   \item Quantify descriptors using the database and get a visual words
%%     index per descriptor
%%   \item Create the histogram with visual words indices
%%   \end{enumerate}

%% \item Inverted index
%%   \begin{enumerate}
%%   \item For all images in the database compute a similarity score between
%%     this image's histogram and the histogram of the queried image
%%   \item Sort the images according to their similarity score in
%%     descending order
%%   \end{enumerate}

%% \item Geometric verification
%%   For the first $N$ images (here $N = 100$)
%%   \begin{enumerate}
%%     \item Search for a good subset of matches that are geometrically
%%       coherent with the queried image (RANSAC) (Geometric
%%       verification)
%%     \item Assign this image the number of geometrically verified
%%       matches as geometric score
%%   \end{enumerate}
%%   Sort the first $N$ images according to their geometric score
%%   (descending order)


%% Return the first $M < N$ images

%% \end{enumerate}

\end{document}
