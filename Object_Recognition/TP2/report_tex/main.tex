\documentclass{article}
%%%%
% Provide the command \fpeval as a copy of the code-level \fp_eval:n.
\usepackage{expl3}[2012-07-08]
\ExplSyntaxOn
\cs_new_eq:NN \fpeval \fp_eval:n
\ExplSyntaxOff
%%%%
\usepackage[english]{babel}
\usepackage[utf8]{inputenc}
\usepackage[margin=1.2in]{geometry}
\usepackage{amsmath}
\usepackage{amsthm}
\usepackage{amsfonts}
\usepackage{amssymb}
\usepackage[usenames,dvipsnames]{xcolor}
\usepackage{graphicx}
\usepackage[siunitx]{circuitikz}
\usepackage{tikz}
\usepackage[colorinlistoftodos, color=orange!50]{todonotes}
\usepackage{hyperref}
\usepackage[numbers, square]{natbib}
\usepackage{fancybox}
\usepackage{epsfig}
\usepackage{soul}
\usepackage[framemethod=tikz]{mdframed}
\usepackage{enumitem}
\usepackage{subcaption}
\usepackage{titlesec}
\usepackage{booktabs}
\input{macros}

%% \setlength{\parskip}{\baselineskip}%
\setlength\parindent{0pt}
\setlength{\parskip}{.5em}

\renewcommand\thesection{\Roman{section}}
\renewcommand\thesubsection{\Roman{section}.\Alph{subsection}}
\titlespacing{\section}
              {0pt}{1\baselineskip}{1\baselineskip}
\titlespacing{\subsection}
              {0pt}{0.8\baselineskip}{0.8\baselineskip}

%%%%%%%%%%%%%%%%%%%%%%%%%%%%%%%%%%%%%%%%%%%
% Header
%%%%%%%%%%%%%%%%%%%%%%%%%%%%%%%%%%%%%%%%%%%

\renewcommand{\assignmenttitle}{Assignment 2: Image classification}
\renewcommand{\studentname}{Rémi Lespinet}
\renewcommand{\email}{remi.lespinet@ens-paris-saclay.fr}

%%%%%%%%%%%%%%%%%%%%%%%%%%%%%%%%%%%%%%%%%%%
% Syntax for using figure macros:
%%%%%%%%%%%%%%%%%%%%%%%%%%%%%%%%%%%%%%%%%%%

% \singlefig{filename}{scalefactor}{caption}{label}
% \doublefig{\subfig{filename}{scalefactor}{subcaption}{sublabel}}
%           {\subfig{filename}{scalefactor}{subcaption}{sublabel}}
%           {global caption}{label}
% \triplefig{\subfig{filename}{scalefactor}{subcaption}{sublabel}}
%           {\subfig{filename}{scalefactor}{subcaption}{sublabel}}
%           {\subfig{filename}{scalefactor}{subcaption}{sublabel}}
%           {global caption}{label}
%
% Tips:
% - with scalefactor=1, a single figure will take the whole page width; a double figure, half page width; and a triple figure, a third of the page width
% - image files should be placed in the image folder
% - no need to put image extension to include the image
% - for vector graphics (plots), pdf figures are suggested
% - for images, jpg/png are suggested
% - labels can be left empty {}

%%%%%%%%%%%%%%%%%%%%%%%%%%%%%%%%%%%%%%%%%%%
% Beginning of assignment
%%%%%%%%%%%%%%%%%%%%%%%%%%%%%%%%%%%%%%%%%%%
\begin{document}
\maketitle


% \singlefig{sift50matches}{0.8}{Representation of $50$ matches  given
%   by the nearest neighbor methods for the two images \emph{all\_souls\_000002.jpg} and
%   \emph{all\_souls\_000015.jpg} for $peakThreshold = 0.005$.}{fig:sift50matches}

% \doublefig{\subfig{similarity_1_3.png}{1}{similarity co-variant detector}{fig:similarity-1-3}}
%         {\subfig{affinity_1_3.png}{1}{affine co-variant detector}{fig:affinity-1-3}}
%         {Comparison of the number of matches between similarity and affine co-variant detectors for low perspective (images 1 and 3)}{fig:affine-1-3}

% \triplefig{\subfig{features-im1-0-0005.png}{1}{$peakThreshold = 0.0005$}{fig:im1-0.0005}}
%           {\subfig{features-im1-0-001.png}{1}{$peakThreshold = 0.001$}{fig:im1-0.001}}
%           {\subfig{features-im1-0-005.png}{1}{$peakThreshold = 0.005$}{fig:im1-0.005}}
%           {Feature detectors with three different values of $peakThreshold$ for image1}{fig:features-im1}

\newpage

\section{Training and testing an Image Classifier}

\subsection{Data preparation and feature extraction}

\question{QA1: Why is the spatial tiling used in the histogram image representation?}

Using spatial tiling adds spatial information about the objects on the
scene, doing so implies to compute K-means 4 times (one for each quadrant),
hence the computed words are 4 times larger.
As a consequence, we will only compare features descriptors that
are in the same quadrant (when we detect a feature, we will compute
its SIFT descriptor, find the bin associated to that descriptor
in the quadrant it has been detected and account for that bin in
the part of the histogram corresponding to that quadrant).

If the number of training images is large enough, this is a good way
to add spatial information without impacting the time complexity
of the classifier.

% Image -> SIFT descriptors -> 4 x K-means sur les descripteurs


% this stems from the fact that we have lost the spatial information procured
% by the tiling

% QE4 We see in the results presented in the question QE2, that the
% histograms normalized with $\mathcal{L}^2$ provided the best results
% compared to non normalized and $\mathcal{L}^1$ normalized histograms


% Using spatial tiling allows us to recognize multiple objects in a scene, to provide
% a spatial information about the objects found in the scene, and to do a more accurate
% histogram matching

% Larger vectors TODO ?

\subsection{Train a classifier for images containing aeroplanes}

\question{QB1: Show the ranked training images in your report}

The ranked images are represented in figure~\ref{fig:rank-airplanes}

\singlefig{rank_airplanes}{0.8}{Representation of the top 36 ranked
  training images classified by a linear SVM classifier on a training
  database composed of aeroplane images (positive) and background
  images (negative)}{fig:rank-airplanes}

\question{QB2: In your report, show relevant patches for the three
  most relevant visual words (in three separate figures) for the top
  ranked training image. Are the most relevant visual words on the
  airplane or also appear on background?}

The three most relevant visual words for the top ranked training image
are represented in figure~\ref{fig:best-airplane-words}.
We can see in this figure that these visual words also appears on the
background, in particular, the rank 1 visual word appears almost
exclusively on the background of the image. An explanation for this
is that images containing airplanes often contains sky, airstrip, etc
which are used by the algorithm to recognize airplanes.

\triplefigc
{\subfig{cropped_airplane_best_word1.png}{0.7}{Rank $1$
    visual word (index $201$) corresponding to $797$ feature
    descriptors, $word\_score = weight \cdot count = 0.21$}
  {fig:best-airplane-word1}}
{\subfig{cropped_airplane_best_word2.png}{0.7}{Rank $2$ visual word
    (index $455$) corresponding to $787$ feature descriptors,
    $word\_score = weight \cdot count = 0.15$}
  {fig:best-airplane-word2}}
{\subfig{cropped_airplane_best_word3.png}{0.7}{Rank $3$ visual word
    (index $336$) corresponding to $871$ feature descriptors,
    $word\_score = weight \cdot count =
    0.14$}
  {fig:best-airplane-word3}}
{Representation of the patches for the 3 most relevant visual words
  for the top ranked image trained on the airplane dataset. count
  number of descriptors associated}{fig:best-airplane-words}

\clearpage

\subsection{Classify the test images and assess the performance}

\question{QC1: Why is the bias term not needed for the image ranking?}

The bias term is not needed because we do not need the exact score, we
only need to rank the images based on their score, and the ranking
goes unchanged if we add an arbitrary constant.

\subsection{Learn a classifier for the other classes and assess its performance}

\question{QD1: In your report, show the top ranked images,
  precision-recall curves and APs for the test data of all the three
  classes (aeroplanes, motorbikes, and persons). Does the AP
  performance for the different classes match your expectations based
  on the variation of the class images?}

% The figure~\ref{fig:rank-test-airplanes} represents the top 36 ranked
% test images in the test airplane set for the classifier
% trained on the train airplane data.

The figure~\ref{fig:rank-test-airplanes},
(resp. \ref{fig:rank-test-motorbikes} and \ref{fig:rank-test-persons})
represents the 36 best ranked test images according to the classifier
trained on the train data using a dataset composed of airplanes
(resp. motorbikes and persons) images and background images.

\singlefig{rank_test_airplanes}{0.82}{top 36 test images classified as airplane by the classifier using a database composed of airplane and background images}{fig:rank-test-airplanes}

\singlefig{rank_test_motorbikes}{0.82}{top 36 test images classified as motorbikes by the classifier using a database composed of motorbikes and background images}{fig:rank-test-motorbikes}

\singlefig{rank_test_persons}{0.82}{top 36 test images classified as persons by the classifier using a database composed of persons and background images}{fig:rank-test-persons}

\clearpage

The precision-recall curves for each of the three classes (airplane,
motorbike and person) is represented in the
figures~\ref{fig:pr-test-airplanes}, \ref{fig:pr-test-motorbikes} and
\ref{fig:pr-test-persons} respectively.

\triplefign
{\subfig{pr_test_airplanes}{0.85}{precision-recall curve for the airplane classe}{fig:pr-test-airplanes}}
{\subfig{pr_test_motorbikes}{0.85}{precision-recall curve for the motorbike classe}{fig:pr-test-motorbikes}}
{\subfig{pr_test_persons}{0.85}{precision-recall curve for the person classe}{fig:pr-test-persons}}
{Representation of the precision-recall curve for each of the three classes}{fig:pr-test}

The AP performance is presented in the table~\ref{tab:ap}
\begin{table}[h!]
  \centering
  \begin{tabular}{| l | l |}
    \hline
    Classe & AP \\
    \hline
    Airplane  & 54.95\% \\
    Motorbike & 48.66\% \\
    Person    & 70.64\% \\
    \hline
  \end{tabular}
  \captionof{table}{AP for each of the 3 classes} \label{tab:ap}
\end{table}

For the three classes, the number of train images is the number of
background train images ($1019$) plus the number of images in the
train data for this class, similarly, the number of test images is the
number of background test images ($1077$) plus the number of images in
the test data for this class. Table~\ref{tab:train-test-comp} shows the
composition of the datasets used for this question on each of the
three classes. (The test ratio corresponds to the random line present
in the precition-recall curves)

\begin{table}[h!]
  \centering
  \begin{tabular}{| c | c | c | c | c | c | c |}
    \hline
    classes & \shortstack{positive \\ train images} & \shortstack{negative \\ train images} & train ratio & \shortstack{positive \\ test images} & \shortstack{negative \\ test images} & test ratio \\
    \hline
    aeroplane & 112 & 1019 & 9.90 \% & 126 & 1077 & 10.47 \%  \\
    motorbike & 120 & 1019 & 10.54 \% & 125 & 1077 & 10.40 \%  \\
    person    & 1025 & 1019 & 50.15 \% & 983 & 1077 & 47.72 \%  \\
    \hline
  \end{tabular}
  \captionof{table}{Composition of the train dataset and the test dataset for each of the 3 classes} \label{tab:train-test-comp}
\end{table}


The performances are correct given the fact that the model is not very
elaborate yet, for the aeroplane dataset, we reach more than 50\%
average precision, which is decent in view of the low fraction number
of images in the test dataset that actually represent an aeroplane
(10\%).

% \doublefig
% {\subfig{rank_test_airplanes}{1}{similarity co-variant detector}{fig:rank-test-airplanes}}
% {\subfig{pr_test_airplanes}{1}{affine co-variant detector}{fig:pr-test-airplanes}}
% {Comparison of the number of matches between similarity and affine co-variant detectors for low perspective (images 1 and 3)}{fig:affine-1-3}


% Similarly, the figures \ref{fig:rank-test-motorbikes} and
% \ref{fig:rank-test-persons} represents the top 36 ranked images
% for the motorbike and person datasets.


\question{QD2: For the motorbike class, give the rank of the first
  false positive image. What point on the precision-recall curve
  corresponds to this first false positive image? Give in your report
  the value of precision and recall for that point on the
  precision-recall curve. }

For the motorbike, the best ranked image is a false positive (it's
represented in figure~\ref{fig:motorbike-test-rank1})

\singlefig{motorbike_test_rank1}{0.4}{Best ranked image for the
  motorbike class (false positive)}{fig:motorbike-test-rank1}

It corresponds to the situation where we only select the top ranked
image. This image is wrong, so the recall is $0$ and the precision is
also $0$. In matlab, if we extract the first 10 points in the precision-recall
curve we obtain
\begin{center}
  \begin{tabular}{| l | l | l | l | l | l | l | l | l | l | l | l }
    \hline
    recall & 0 & 0 & 0.008 & 0.016 & 0.024 & 0.032 & 0.04 & 0.040 & 0.040 & 0.048 & \dots \\
    \hline
    precision & 1 & 1.0e-10 & 0.500 & 0.667 & 0.750 & 0.800 & 0.833 & 0.714 & 0.625 & 0.667 & \dots \\
    \hline
  \end{tabular}
\end{center}
The corresponding point is the second entry ($0$, $1.0e-10$).  The
first point is a convention when no sample are predicted (the
number of positive images returned as well as the number of return
images is $0$, so the precision is $0$ over $0$ which is $1$ by
convention, and the number of positive images is $120$ for the
motorbike, so the recall is $0$ over $120$ which is $0$).
We can see that point in the figure~\ref{fig:pr-test-motorbikes}.


\subsection{Vary the image representation}

\question{QE1: Include in your report precision recall-curves and APs,
  and compare the test performance to the spatially tiled
  representation in stage D. How is the performance changing? Why?}

The precision-recall curves for all datasets are presented in
figure~\ref{fig:pr-test-notiling}. The comparison of the average
precision with and without spatial tiling is presented in the
table~\ref{tab:ap-notiling}. We see that the performance is
decreasing, this stems from the fact that we have lost the spatial
information provided by the tiling. (For the next questions, I
have reactivated the spatial tiling)

\triplefign
{\subfig{pr_test_airplanes_notiling}{0.85}{precision-recall curve for the airplane classe}{fig:pr-test-airplanes-notiling}}
{\subfig{pr_test_motorbikes_notiling}{0.85}{precision-recall curve for the motorbike classe}{fig:pr-test-motorbikes-notiling}}
{\subfig{pr_test_persons_notiling}{0.85}{precision-recall curve for the person classe}{fig:pr-test-persons-notiling}}
{Precision-recall curve for each of the three classes when there's no spatial tiling}{fig:pr-test-notiling}

\begin{table}[h!]
  \centering
  \begin{tabular}{| c | c | c |}
    \hline
    Classe & AP (tiling) & AP (no tiling) \\
    \hline
    Motorbike & \textbf{48.66\%} & 41.54\%\\
    Airplane  & \textbf{54.95\%} & 51.82\% \\
    Person    & \textbf{70.64\%} & 69.61\%\\
    \hline
  \end{tabular}
  \captionof{table}{Effect of tiling on average precision for each of
    the 3 classes} \label{tab:ap-notiling}
\end{table}

\question{QE2: Modify exercise1.m to use L1 normalization and no
  normalization and measure the performance change}

The table~\ref{tab:ap-normalization} presents the different average
precision obtained for different type of normalizations
($\mathcal{L}^2$, $\mathcal{L}^1$ and no normalization) for each of
the three classes (Airplane, Motorbike and Person).

\begin{table}[h!]
  \centering
  \begin{tabular}{| c | c | c | c |}
    % \hline
    % & \multicolumn{3}{c|}{Average precision (AP)} \\
    \hline
    Classe & $\mathcal{L}^1$ normalization & $\mathcal{L}^2$ normalization & no normalization \\
    \hline
    Motorbike & 26.00\% & 48.66\% & \textbf{48.73\%} \\
    Aeroplane & 51.68\% & 54.95\% & \textbf{62.45\%} \\
    Person    & 56.54\% & \textbf{70.64\%} & 67.20\% \\
    \hline
  \end{tabular}
  \captionof{table}{Effect of the histogram normalization on average
    precision for each of the 3 classes} \label{tab:ap-normalization}
\end{table}

% \begin{table}[h!]
%   \centering
%   \begin{tabular}{| c | c | c | c |}
%     \hline
%     Classe & $\mathcal{L}^1$ normalization & $\mathcal{L}^2$ normalization & no normalization \\
%     \hline
%     Airplane  & \textbf{56.24\%} & 51.82\% & 55.79\%\\
%     Motorbike & 31.43\% & \textbf{41.54\%} & 32.16\%\\
%     Person    & 60.14\% & \textbf{69.61\%} & 63.79\%\\
%     \hline
%   \end{tabular}
%   \captionof{table}{Effect of the histogram normalization on average precision for each of
%     the 3 classes (no spatial tiling) } \label{tab:ap-normalization}
% \end{table}

\question{QE3: What can you say about the self-similarity, K(h,h), of
  a BoVW histogram h that is L2 normalized?}

We have
\begin{equation*}
  K(h, h) = \sum_{i = 1}^d h_i^2 = \| h \|_2^2
\end{equation*}
Hence if h is $\mathcal{L}^2$ normalized, $K(h, h) = 1$

If $h$ and $h'$ are $\mathcal{L}^2$ normalized, Cauchy-Schwarz
inequality gives us
\begin{equation*}
  K(h, h') \le K(h, h) K(h', h') = 1
\end{equation*}
This means that $K(h, h') \le 1$ with equality when $h = h'$.

This is not true if we normalize with $\mathcal{L}^1$, it is possible
to have $K(h, h') > K(h, h)$ with $h$ and $h'$ being different.
For example in dimension 2, if we take $h = (0.75, 0.25)$ and
$h' = (1, 0)$, we have $\|h\|_1 = 1$, $\|h'\|_1 = 1$,
$K(h, h') = 0.75$, but $K(h, h) = 0.625$


\question{QE4: Do you see a relation between the classification performance and L2 normalization?}

We can see in the table that the $\mathcal{L}^2$ gives better
results than the $\mathcal{L}^1$ norm for all the 3 classes, the
results on this training set if we don't normalize are slighly
better in the \emph{Motorbike} and \emph{Aeroplane} cases.

% results for the class \emph{person},  when compared to the
% $\mathcal{L}^1$ norm and to no normalization at all.

\subsection{Vary the classifier}

\question{QF1: Based on the rule of thumb introduced above, how should
  the BoVW histograms h and h' be normalized? Should you apply this
  normalization before or after taking the square root?}

Explicitely computing the mapping means taking the square root of the
histogram values $\tilde{h}_i = \sqrt{h_i}$. If we feed the linear SVM
classifier with $\tilde{h}$, from the proposed rule of thumb we want
$\tilde{h}$ to be $\mathcal{L}^2$ normalized.

\begin{equation*}
  h_i \xrightarrow{\sqrt{.}} \sqrt{h_i} \xrightarrow{\mathcal{L}^2 normalize} \dfrac{\sqrt{h_i}}{\sqrt{\sum_{i = 1}^d \sqrt{h_i}^2}} = \dfrac{\sqrt{h_i}}{\sqrt{\sum_{i = 1}^d |h_i|}}
\end{equation*}

This is equivalent to
\begin{equation*}
  h_i \xrightarrow{\mathcal{L}^1 normalize} \dfrac{h_i}{\sum_{i = 1}^d |h_i|} \xrightarrow{\sqrt{.}} \dfrac{\sqrt{h_i}}{\sqrt{\sum_{i = 1}^d |h_i|}}
\end{equation*}

So we can either normalize $h$ in $\mathcal{L}^1$ before applying the
square root or take the square root and then normalize in $\mathcal{L}^2$

\question{QF2: Why is this procedure equivalent to using the Hellinger
  kernel in the SVM classifier?}

Given two histograms $h$ and $h'$, the given procedure will feed the
linear SVM with the modified input histograms $\hat{h}$ and $\hat{h'}$,
and we have

% This is equivalent, because the linear SVM will be measuring similarities
% using the modified input histograms $\hat{h}$, and the
\begin{equation*}
  K_{Linear}(\hat{h}, \hat{h'}) = \sum_{i = 1}^d \hat{h}_i \hat{h'}_i = \sum_{i = 1}^d \sqrt{h_i} \sqrt{h'_i} = K_{Hellinger}(h, h')
\end{equation*}

Hence this procedure is equivalent to using the Hellinger kernel in
the SVM classifier.

\question{QF3: Why is it an advantage to keep the classifier linear,
  rather than using a non-linear kernel?}

Using a linear method on the modified histograms allows to compute
the square roots only once, rather than computing it each time
the kernel is evaluated, which increases performance.

\question{QF4: Try the other histogram normalization options and check
  that your choice yields optimal performance. Summarize your finding
  in the report (include only mAP results, no need to include the full
  precision-recall curves).}

The comparison of average precisions for different normalizations
($\mathcal{L}^1$, $\mathcal{L}^2$ and no normalization) of the
histogram, using a Hellinger kernel classifier are presented in the
table~\ref{tab:ap-normalization-hellinger}. We see that, as expected,
the overall best results are obtained using $\mathcal{L}^1$
normalization.

% \begin{table}[h!]
%   \centering
%   \begin{tabular}{| c | c | c | c | c |}
%     \hline
%     class & $\mathcal{L}^2_{before}$ & $\mathcal{L}^1_{before}$=$\mathcal{L}^2_{after}$ & $\mathcal{L}^1_{after}$ & none \\
%     \hline
%     Motorbike & 49.982\% & 49.165\% & 26.805\% & 44.009\% \\
%     Aeroplane & 56.226\% & 67.637\% & 59.960\% & 56.199\% \\
%     Person    & 71.164\% & 74.778\% & 59.425\% & 66.431\% \\
%     \hline
%   \end{tabular}
%   \captionof{table}{Effect of the histogram normalization on average precision for each of
%     the 3 classes with the Hellinger kernel } \label{tab:ap-normalization-hellinger}
% \end{table}

\begin{table}[h!]
  \centering
  \begin{tabular}{| c | c | c | c | c |}
    \hline
    class & $\mathcal{L}^1$ & $\mathcal{L}^2$ & none \\
    \hline
    Motorbike & \textbf{63.25} & 52.81 & 55.64 \\
    Aeroplane & \textbf{70.72} & 63.94 & 65.87 \\
    Person    & \textbf{77.39} & 71.45 & 69.71 \\
    \hline
  \end{tabular}
  \captionof{table}{Effect of the histogram normalization on average precision for each of
    the 3 classes with the Hellinger kernel } \label{tab:ap-normalization-hellinger}
\end{table}

We also see that the Hellinger using the $\mathcal{L}^1$ norm yields
better results than the Linear kernel using the $\mathcal{L}^2$ norm
for each of the 3 classes. We obtain about $15\%$ increase in average
precision for the motorbike and the aeroplane, and about $7\%$ for the
person class, which is a significative improvement. (The
table~\ref{tab:encodings} present these results as well as the results
for different encoding, namely VLAD and FV that we try later on)

% \begin{table}[h!]
%   \centering
%   \begin{tabular}{| c | c | c | c |}
%     \hline
%     Classe & $\mathcal{L}^1$ normalization & $\mathcal{L}^2$ normalization & no normalization \\
%     \hline
%     Airplane  & \textbf{67.64\%} & 56.23\% & 56.20\% \\
%     Motorbike & 49.17\% & \textbf{49.99\%} & 44.01\% \\
%     Person    & \textbf{74.78\%} & 71.16\% & 66.43\% \\
%     \hline
%   \end{tabular}
%   \captionof{table}{Effect of the histogram normalization on average precision for each of
%     the 3 classes with the Hellinger kernel (no spatial tiling) } \label{tab:ap-normalization-hellinger}
% \end{table}

\subsection{Vary the number of training images}

\question{QG1: Report and compare performance you get with the linear
  kernel and with the Hellinger kernel for the different classes and
  proportions of training images (10\%, 50\% and 100\%). You don’t
  have to report the precision-recall curves, just APs are
  sufficient. Plot three curves (one curve for each
  class) into one figure. Produce two figures, one for the linear
  kernel and one for the Hellinger kernel. Make sure to properly label
  axis (use functions xlabel and ylabel), show each curve in a
  different color, and have a legend (function legend) in each
  figure. Show the two figures in your report.}

The table~\ref{tab:ap-ratios} present a comparison of the average
precision obtained for different values of the ratio of training
images taken (10\%, 50\%, 100\%) using a linear and a Hellinger kernel.

The average precision increases with the number of training images as
we can expect, we see that the Hellinger kernel gives overall better
results, and seems to work better when there the number of training
images is low.

\begin{table}[h!]
  \centering
  \begin{tabular}{| c | c | c | c || c | c | c |}
    \hline
    & \multicolumn{3}{c||}{Linear kernel} & \multicolumn{3}{c|}{Hellinger kernel} \\
    \hline
    class & $r = 0.1$ & $r = 0.5$ & $r = 1$ & $r = 0.1$ & $r = 0.5$ & $r = 1$ \\
    \hline
    motorbike  & 25.23\% & 38.23\% & \textbf{48.66\%} & 36.85\% & 54.74\% & \textbf{63.25\%} \\
    aeroplane  & 32.77\% & 40.50\% & \textbf{54.95\%} & 54.05\% & 64.44\% & \textbf{70.72\%} \\
    person     & 59.89\% & 67.23\% & \textbf{70.64\%} & 66.86\% & 73.26\% & \textbf{77.39\%} \\
    \hline
  \end{tabular}
  \captionof{table}{Comparison of the average precision for different
    values of proportions of training images ($r$) for each of the 3
    classes for the linear and Hellinger kernel
  } \label{tab:ap-ratios}
\end{table}


% \begin{table}[h!]
%   \centering
%   \begin{tabular}{| c | c | c | c |}
%     \hline
%     class & $r = 0.1$ & $r = 0.5$ & $r = 1$ \\
%     \hline
%     motorbike  & 25.23\% & 38.23\% & \textbf{48.66\%} \\
%     aeroplane  & 32.77\% & 40.50\% & \textbf{54.95\%} \\
%     person     & 59.89\% & 67.23\% & \textbf{70.64\%} \\
%     \hline
%   \end{tabular}
%   \captionof{table}{Average precision for different values of
%     proportions of training images ($r$) for each of the 3 classes
%     using a Linear kernel } \label{tab:ap-ratio-linear}
% \end{table}

% \begin{table}[h!]
%   \centering
%   \begin{tabular}{| c | c | c | c |}
%     \hline
%     class & $r = 0.1$ & $r = 0.5$ & $r = 1$ \\
%     \hline
%     motorbike  & 36.85\% & 54.74\% & \textbf{63.25\%} \\
%     aeroplane  & 54.05\% & 64.44\% & \textbf{70.72\%} \\
%     person     & 66.86\% & 73.26\% & \textbf{77.39\%} \\
%     \hline
%   \end{tabular}
%   \captionof{table}{Average precision for different values of
%     proportions of training images ($r$) for each of the 3 classes
%     using a Hellinger kernel } \label{tab:ap-ratio-hellinger}
% \end{table}

The figure~\ref{fig:ap-ratio} presents the variation of the average
precision as a function of the percentage of training images taken for
the training part for each of the three classes for both the linear
(figure~\ref{fig:ap-ratio-linear}) and the Hellinger kernel
(figure~\ref{fig:ap-ratio-hellinger}).

\doublefig
{\subfig{ap_ratio_linear}{1.1}{Linear
    kernel}{fig:ap-ratio-linear}}
{\subfig{ap_ratio_hellinger}{1.1}{Hellinger
    kernel}{fig:ap-ratio-hellinger}} {Average precision as a function
  of the percentage of the training images taken for training for each
  of the three classes (20 points are represented)}{fig:ap-ratio}

\question{QG2: By analyzing the two figures, do you think the
  performance has `saturated' if all the training images are used, or
  would adding more training images give an improvement?}

We see in the curves that the average precision increases with the
number of training, especially when the number of sample is low
(between $10\%$ and $70\%$ of the training images taken). When
approaching a ratio of 1, the curve still seems to increase even if
it's very slight, hence, I think that adding images could make our
average precision gain fractions of percents, but it will not make a
huge improvement (especially for the \emph{person} class which already
has a large amount of training images).

% Given the curves, I think adding training samples will potentially
% result in a small gain,

% I don't think reaching

% I dont think the performance is completely \emph{saturated}, at least
% for the \emph{motorbike} and the \emph{aeroplane}, there is still a
% gain in taking $100\%$ of the sample versus $90\%$,

% curve seem to
% still increase even when we take $90\%$ of the training samples,
% especially for the Hellinger kernel, which is less noisy


% The curves of the average precision for the \emph{motorbike} and the
% \emph{aeroplane} do not seem to be saturated, they are still increasing
% when we take $90\%$ of the training images
% when we take $80\%$ of
% the training sample, their respective values of average precision are
% $43.92$ and


% TODO

\section{Training an Image Classifier for Retrieval using Internet image search}

\question{QP2.1: For the horse class, report the precision at rank-36
  for 5 and 10 training images. Show the training images you used. Did
  the performance of the classifier improve when 10 images were used?}

I downloaded $28$ images and trained the classifier $1000$ times over
5 random images from this set of $28$ images, I obtained the histogram
presented on figure~\ref{fig:hist-top-5-horses}. I did the same for
$10$ over $28$ images, the histogram is presented on figure
\ref{fig:hist-top-10-horses}. The table~\ref{tab:stats-horses} contains
statistics for both the selection of $5$ training images and $10$
training images.

\doublefig
{\subfig{hist_top_5_horses}{1.1}{5 images selected}{fig:hist-top-5-horses}}
{\subfig{hist_top_10_horses}{1.1}{10 images selected}{fig:hist-top-10-horses}}
{ Histograms representing the number of good classified images in the
  top 36 scoring images for sets of training images containing 5 and
  10 images (on 28 images downloaded)}{fig:hist-top-horses}

\begin{table}[h!]
  \centering
  \begin{tabular}{| c | c | c | c | c | c | c |}
    \hline
    \begin{tabular}{c}
      selected images \\
      (over 28) \\
    \end{tabular}
    & $AP_{mean}$ & $AP_{max}$ & $AP_{std}$ & $Top\_36_{mean}$ & $Top\_36_{max}$ & $Top\_36_{std}$\\
    \hline
    % 5 & $15.64$\% & $40.54$\% & $5.56$\% & $7.99$ & $24$ & $5.04$ \\
    5 & $17.04$\% & $45.68$\% & $8.03$\% & $8.54$ & $27$ & $6.95$ \\
    % 10  & $37.07$\% & $58.12$\% & $8.26$\% & $20.97$ & $32$ & $4.96$ \\
    % 10  & $35.89$\% & $52.13$\% & $7.39$\% & $20.49$ & $32$ & $4.55$ \\
    % 10  & $36.60$\% & $51.56$\% & $7.67$\% & $20.84$ & $30$ & $4.71$ \\
    10  & $36.19$\% & $54.30$\% & $7.50$\% & $20.68$ & $30$ & $4.68$ \\
    \hline
  \end{tabular}
  \captionof{table}{Statistics obtained by the procedure described} \label{tab:stats-horses}
\end{table}

The 5 training images that produced the highest number of good
classified images in the top 36 are represented in the
figure~\ref{fig:horses-5-training} and the top 36 images returned by
the classifier are shown in figure~\ref{fig:top-horses-5-training}
. Similarly, the set of 10 training images that produced the best
result is represented in the figure~\ref{fig:horses-10-training}, and
the top 36 images in the figure~\ref{fig:top-horses-10-training},

\singlefig{horses_5_training}{0.85}{Set of $5$ training images of
  horses downloaded on the internet used to train the classifier}{fig:horses-5-training}

\singlefig{top_horses_5_training}{0.85}{top 36 images classified as
  horses by the classifier trained on a database composed of 5 images
  downloaded from the internet (represented in
  fig~\ref{fig:horses-5-training}) $27/36$ are correctly
  classified}{fig:top-horses-5-training}

\singlefig{horses_10_training}{0.85}{Set of $10$ training images of
  horses downloaded on the internet used to train the classifier}{fig:horses-10-training}

\singlefig{top_horses_10_training}{0.85}{top 36 images classified as
  horses by the classifier trained on a database composed of 10 images
  downloaded from the internet (represented in
  fig~\ref{fig:horses-10-training}). $30/36$ are correctly
  classified}{fig:top-horses-10-training}

The performance of the classifier is clearly improved when $10$ images
are used instead of $5$, the histogram is shifted right, the mean
precision in the top36 increases from about $8$ (22\%) to $21$ over
$36$ (58\%) which is very significiant.

\clearpage

\question{QP2.2: What is the best performance (measured by precision
  at rank-36) you were able to achieve for the horse and the car
  class? How many training images did you use? For each of the two
  classes, show examples of your training images, show the top ranked
  36 images, and report the precision at rank-36. Compare the
  difficulty of of retrieving horses and cars}

The best performance I've reached for the horse class is $32$ achieved
with a set of $13$ images, e.g $88.89\%$ precision at
rank 36 (see figure~\ref{fig:horses-13-training} for the training images
and figure~\ref{fig:top-horses-13-training} for the top 36
ranked images)

For the car class, I reached $36$ with $7$ images ($100\%$ precision at
rank 36). The figure~\ref{fig:cars-7-training} shows the training images and the
figure~\ref{fig:top-cars-7-training} shows the top 36 ranked images
from the test dataset.

\singlefig{horses_13_training}{0.80}{Set of $13$ training images of
  horses used to reach $32$ good classified images in the top 36
  ranked images (best I could achieve)}{fig:horses-13-training}

\singlefig{top_horses_13_training}{0.80}{top 36 images classified as horses
  by the classifier trained on a database composed of 7 images
  downloaded from the internet and background images (false
  positive)}{fig:top-horses-13-training}

\singlefig{cars_7_training}{0.7}{Set of $7$ training images of cars
  used to reach $36$ good classified images in the top rank-36}{fig:cars-7-training}

\singlefig{top_cars_7_training}{0.85}{top 36 images classified as cars
  by the classifier trained on a database composed of 7 images
  downloaded from the internet and background images (false
  positive)}{fig:top-cars-7-training}

We can notice that recognizing horses is a lot more difficult than
recognizing cars, for the car, reaching $100\%$ precision at rank 36
was actually fairly easy. This can be explained by the fact that there
is a lot more recognizable detection points on a car than on a horse
(due to the sharp geometry of the object, the different material used,
\dots)

\clearpage

\section{Advanced Encoding Methods}

\subsection{First order methods}

\question{QH1: Compare the dimension of VLAD and BoVW vectors for a
  given value of K. What should be the relation of the K in VLAD to
  the K in BoVW in order to obtain descriptors of the same dimension?
  You can ignore tiling.}

The dimension of the VLAD vector is $K N$, K being the size of
the vocabulary and $N$ the size of a SIFT vector.
The dimension of a BoVW vector is $K$, with K the size of the
vocabulary.

For a BoVW vector to have the same dimension as a VLAD vector which
has a vocabulary of size $K$, we need a vocabulary that is $N$ times
bigger than the one used for the VLAD vector e.g a vocabulary of size
$K'$ with $K' = K N$.

\question{QH2: Replace the encoding used in exercise1 with the VLAD
encoding, and repeat the classification experiments for the three
classes of Part I (Both linear and Hellinger kernel). How do the
results compare to the BoVW encoding? Report AP results in a
table. No need to report all precision-recall curves.}

By looking at the dimensions in MATLAB, we see that the vectors are of
size $10240$, versus $2048 = 4 \cdot 512$ (512 per tile) for the BoVW
vectors, assuming there's also tiling, this means that $K = 20$,
$N = 128$.

% \begin{table}[h!]
%   \centering
%   \begin{tabular}{| c | c | c | c |}
%     \hline
%     classes & BoVW & VLAD & FV \\
%     \hline
%     Motorbike & 48.05\% & \textbf{68.36\%} & 69.17\\
%     Aeroplane & 54.61\% & \textbf{70.89\%} & 66.75\\
%     Person    & 70.61\% & \textbf{77.27\%} & 78.62\\
%     \hline
%   \end{tabular}
%   \captionof{table}{Comparison of BoVW and VLAD encoding method for each of
%     the 3 classes with the linear kernel} \label{tab:encoding-linear}
% \end{table}

We need to be extra-careful, because a VLAD vector can have negative
components. To use Hellinger's kernel, we use signed square-rooting
(for each component we take the square root the absolute value and
multiply by its sign) this is appropriate because in the
formula below K is a measurement of similarity, and doing this
operation ensures that if two multiplied components have the same sign
(which means that they are similar) $K$ is increased, whereas if they
have opposite sign (which means they are not similar), $K$ is
decreased.
\begin{equation*}
  K(h, h') = \sum_{i = 1}^d \sqrt{h_i h'_i}
\end{equation*}

The results of the average precision for each of the three classes
using the VLAD encoding are represented in the
table~\ref{tab:encodings}. We can see that the classifier using VLAD
encoding outperforms the one using BoVW encoding for the two kernel
used. The increase in average precision is particularly good for the
motorbike and aeroplane cases (about $20\%$ increase for the linear
kernel).

\begin{table}[h!]
  \centering
  \begin{tabular}{| c | c | c | c || c | c | c |}
    \hline
    & \multicolumn{3}{c||}{Linear kernel} & \multicolumn{3}{c|}{Hellinger kernel} \\
    \hline
    classes & BoVW & VLAD & FV & BoVW & VLAD & FV \\
    \hline
    motorbike  & 48.66\% & 68.82\% & \textbf{72.67\%} & 63.25\% & 75.42\% & \textbf{81.14\%} \\
    aeroplane  & 54.95\% & \textbf{74.62\%} & 70.64\% & 70.72\% & 75.56\% & \textbf{78.13\%} \\
    person     & 70.64\% & 75.54\% & \textbf{77.44\%} & 77.39\% & 78.86\% & \textbf{82.11\%} \\
    \hline
  \end{tabular}
  \captionof{table}{Comparison of BoVW and VLAD encoding method for
    each of the 3 classes with the linear and Hellinger
    kernel} \label{tab:encodings}
\end{table}


% \begin{table}[h!]
%   \centering
%   \begin{tabular}{| c | c | c | c |}
%     \hline
%     classes & BoVW & VLAD & FV\\
%     \hline
%     Motorbike & 49.17\% & \textbf{70.33\%} & 78.25\\
%     Aeroplane & 67.64\% & \textbf{70.86\%} & 77.20\\
%     Person    & 74.78\% & \textbf{76.96\%} & 79.14\\
%     \hline
%   \end{tabular}
%   \captionof{table}{Comparison of BoVW and VLAD encoding method for each of
%     the 3 classes with the Hellinger kernel} \label{tab:encoding-hellinger}
% \end{table}

\subsection{Second order methods}

\question{QI1: Replace the encoding used in exercise1 with the FV
  encoding, and repeat the classification experiments for the three
  classes of Part I. Report the results in the same table as QH2 so
  that you can see the performance of the three encoding methods side
  by side.}

The results are presented in the previous table
(table~\ref{tab:encodings}) for the linear and the Hellinger kernel
classifier)

\question{QI2: What are the advantages or disadvantages of FV compared
  to VLAD in terms of computation time and storage/memory footprint -
  especially for a large number (hundreds of millions) of images.}

We have seen that a VLAD vector has $K D$ components and that A Fisher
Vector has $2 K D$ component in the case where only the variance of
each component is computed. This is actually how it's done in our case
(only the variance of each component is computed), if we look at the
amount of spaces taken on disk to store a fisher vector and a vlad
vector, we see that the fisher vector takes about twice the size of
the vlad vector (for example \emph{background\_train\_vlad.mat} takes
38132 Ko and \emph{background\_train\_fv.mat} takes 74384 Ko).

If we would want to compute the covariance matrix entirely, this would
be $K D + K D^2$. In our case, $D = 128$, if we take $K = 20$, this
number is already $330240$ components per image, multiplied by
$100 000 000$ images this is about $132TB$ if we count $4$ bytes per
components.

If we only store the diagonal, we only have $5120$ components per
image, which gives $2TB$ of data. VLAD vectors would have given half
this amount, which is $1TB$ of data.

In the processing part, the VLAD method need to compute the SIFT
descriptors for each image and run K-means algorithm. For each image
and for each descriptor, it then needs to compute the sum of
residual. We then learn a classifier (SVM).

At run time, we need to find the centroid associated to each
descriptor, using a kd-tree, compute the residual with respect to this
centroid and account for it to the VLAD vector, once we have done that
for all descriptors, we can predict using the classifier.

the FV method need to compute the SIFT descriptor for each image, and
run GMM algorithm. For each image and for each descriptor, it then
computes the mean of the gaussian kernel with the maximum probability
(hopefully we can do that efficiently using an accelerating structure
(some kind of modified kd-tree)). We then need to compute the residual
and the variance (which we can do in one pass). once we have done that
for each descriptors, and for each image, we learn a classifier.

At run time, for each descriptor, we need to find the mean of the
gaussian kernel associated to the maximum probability, compute the
residual, and the variance (we can do that online), we can then
predict using the classifier.

We can notice that the GMM clustering is slower than K-means, assuming
that there is an efficient structure for finding the mean of the
gaussian kernel associated to the maximum probability, this part is
slower by only a small amount, and for the FV, we only need to compute
the variance, which should increase the time by a small amount at run
time (but this cost can be important in preprocessing as it is multiplied
by the number of training images).

% \begin{equation*}
%   \underset{N}{\text{Images}} \xrightarrow{SIFT} \underset{N \times D \times M}{\text{Descriptors}} \xrightarrow{K-means} \underset{K}{\text{Centroids}}
% \end{equation*}



\end{document}
